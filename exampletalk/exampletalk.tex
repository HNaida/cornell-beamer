\documentclass[serif]{beamer}

%set the theme to Cornell and set options.
%navbar=true shows the navigation bar in the footline. navbar=false hides it
%colorblocks=true makes the block (and theorem) environment appear as a colored box. colorblocks=false makes the block (and theorem) environment very plain.
\mode<presentation>
{
\usetheme
[navbar=true,colorblocks=true,pagenumbers=true]{Cornell}
}

%these packages are essential for compiling
\usepackage[english]{babel}
\usepackage[T1]{fontenc}
\usepackage{lmodern}

%short title appears in headline, long title appears on title page, subtitle appears on title page
\title[Short Title]{Long Title}
\subtitle{Subtitle}

%the author only appears in the headline of slides
\author[]{Author Name}

%the institute contains all information, including author name. only appears on title page
\institute
{
\begin{tabular}[h]{cc}
\normalsize Author One  & \normalsize Author Two \\
~\\
Institute One           & Institute Two \\
Department One          & Department Two \\
{\tt email@one.com}     & {\tt email@two.com}
\end{tabular}
}

\date[]{October 11, 2011}

%uncomment the following lines to change major colors in the theme. they are currently set to their defaults.

%\setbeamercolor*{structure}{fg=cblue} %misc. elements, like toc pages and itemize
%\setbeamercolor{palette secondary}{fg=cred} %footline
%\setbeamercolor{palette tertiary}{fg=white,bg=cgray} %headline
%\setbeamercolor{palette quaternary}{fg=cred} %title
%\setbeamercolor{high stripe}{bg=cred} %stripe in title
%\definecolor{block color}{named}{cblue} %normal block colors


\begin{document}

%#################################################
\begin{frame}[plain]
\titlepage
\end{frame}
%#################################################

\part{Basic Usage}
\section{Good and Bad}
\subsection*{}

%#################################################
\begin{frame}{The Good}
\begin{itemize}
\item \textbf{LOTS} of screen space on each slide.
\item Vector graphics: scales to any screen size.
\item Has a reasonably well integrated color scheme.
\item Navigation bar can be micro-managed.
\item Some options: navigation bar and block environment.
\end{itemize}
\end{frame}
%#################################################

%#################################################
\begin{frame}{The Bad}
Kind of fiddly.
\begin{itemize}
\item Small template changes (font, text size, different spacings, etc.) may break entire theme.
\item ``Bad box'' exceptions when compiling are possible.
\end{itemize}
Requires non-standard inputs.
\begin{itemize}
\item Empty subsections (or non-empty, if you prefer) required after each section change for minipages to appear in navigation bar.
\item Need to format the ``institute'' in a non-standard way.
\item Need to put main author's name in institute section.
\end{itemize}
\end{frame}
%#################################################

\section{Using Theme}
\subsection*{}

%#################################################
\begin{frame}{Title, Author, Date}
Use the annotated TeX file to learn the necessary inputs prior to the first slide, like:
\begin{itemize}
\item The name to appear in the header.
\item The title to appear in the header.
\item The information to appear on the title slide.
\end{itemize}
\end{frame}
%#################################################

%#################################################
\begin{frame}{Options}
There are three options in the theme:
\begin{enumerate}
\item navbar (set to true in these slides)
\begin{itemize}
\item Setting to true displays navigation bar in the footline.
\item Setting to false hides navigation bar in the footline.
\end{itemize}
\item colorblocks (set to true in these slides)
\begin{itemize}
\item Setting to true makes block (and theorem) environment appear as colored box.
\item Setting to false makes block (and theorem) environment very plain.
\end{itemize}
\item pagenumber (set to true in these slides)
\begin{itemize}
\item Setting to true displays page numbers in the footline.
\item Setting to false hides page numbers in the footline.
\end{itemize}
\end{enumerate}
\end{frame}
%#################################################

%#################################################
\begin{frame}{Table of Contents}
Table of contents slides are nice.
\begin{itemize}
\item Learn to use them: page 100 in Beamer manual.
\item You probably don't want to display subsections.
\end{itemize}

\end{frame}
%#################################################

%#################################################
\tableofcontents[currentsection,subsectionstyle=hide]
%#################################################


\section{Navigation Bar}
\subsection{}

%#################################################
\begin{frame}{Sections, Subsections, and Parts}

The navigation bar is small. This may force you to use parts!
\begin{itemize}
\item Learn to use them: page 98 in Beamer manual.
\item Sections in different parts don't display in the same navigation bar. This eliminates crowded navigation bars.
\item If you have many sections split them into multiple parts.
\item If you have many slides split them into multiple parts.
\end{itemize}

\end{frame}
%#################################################

%#################################################
\begin{frame}{Navigation Bar}

\begin{itemize}
\item You can turn off the navigation bar via the navbar option.
\end{itemize}

\begin{block}{If you want just sections in the navigation bar:}
\begin{itemize}
\item Create sections.
\item Don't create subsections.
\end{itemize}
\end{block}

\begin{block}{If you want sections and a dot underneath for each slide:}
\begin{itemize}
\item Create sections.
\item If you want a slide to appear as a dot put it in a subsection.
\item You can make empty subsections with \textbackslash subsection\{\}.
\end{itemize}
\end{block}

\end{frame}
%#################################################


\section{Dummy Section}

%#################################################
\begin{frame}{Dummy Slide}

This slide does not appear as a dot in the navigation bar because it is not in a subsection.

\end{frame}
%#################################################


\section{Special Text}
\subsection{}

%#################################################
\begin{frame}{Blocks}

\begin{block}{Blocks}
are grayish blue. Itemize and enumerate items are unchanged in blocks (and theorems).
\begin{itemize}
\item Item color.
\end{itemize}
\begin{enumerate}
\item Enumerate color.
\end{enumerate}
\end{block}


\end{frame}
%#################################################


%#################################################
\begin{frame}{Special Text}

\begin{alertblock}{Alerted Blocks}
and alerted text are Cornell red. Itemize and enumerate items change color in alerted blocks.
\begin{itemize}
\item Item color.
\end{itemize}
\begin{enumerate}
\item Enumerate color.
\end{enumerate}
\end{alertblock}

\begin{exampleblock}{Example Blocks}
and example text are grayish green. Who cares anyway, no one uses these. Itemize and enumerate items change color in example blocks.
\begin{itemize}
\item Item color.
\end{itemize}
\begin{enumerate}
\item Enumerate color.
\end{enumerate}
\end{exampleblock}

\end{frame}
%#################################################

\section{Colors}
\subsection{}

%#################################################
\begin{frame}{Beamer Colors}

There are a few Beamer colors that can be tampered with. The code for changing them is in the preamble of this TeX file.
\begin{itemize}
\item structure: controls misc. elements, like colors in toc pages and itemize elements.
\item palette secondary: controls colors in footline.
\item palette tertiary: controls colors in headline.
\item palette quaternary: controls colors in slide titles.
\item high stripe: controls color of stripe under slide titles.
\end{itemize}
~\\
There is one non-beamer color of interest:
\begin{itemize}
\item block color: controls colors in regular blocks.
\end{itemize}

\end{frame}
%#################################################

%#################################################
\begin{frame}{Color Definitions}

There are many custom colors used in this template. Each of them is a prefixed with a ``c'' for cornell.
\begin{itemize}
\item \textbf{\textcolor{cred}{cred}}: RGB 179,27,27
\item \textbf{\textcolor{cblue}{cblue}}: RGB 79,111,138
\item \textbf{\textcolor{cgray}{cgray}}: RGB 104,100,91
\item \textbf{\textcolor{cgreen}{cgreen}}: RGB 30,119,72
\end{itemize}
~\\
There are other colors that are defined and unused:
\begin{itemize}
\item \textbf{\textcolor{cpurple}{cpurple}}: RGB 128,66,128
\item \textbf{\textcolor{cyellow}{cyellow}}: RGB 212,170,0
\item \textbf{\textcolor{corange}{corange}}: RGB 206,102,0
\end{itemize}
~\\
Using only these in accompanying images will keep everything looking uniform.


\end{frame}
%#################################################

\part{Future Work}

\section{Future Work}

%#################################################
\begin{frame}{Future Work}
There are two somewhat complicated things to implement:
\begin{itemize}
\item Change the style of the block environment
\item Create a TOC for parts similar to the one that exists for sections
\end{itemize}
There is also some easier but time consuming stuff to implement:
\begin{itemize}
\item Make logos Tikz drawings and embed them in the style file
\item Include an option to put the Cornell logo in the top corner on every slide. Currently available in the style file by uncommenting
\item An option to make each subsection appear as a dot in the navbar instead of each page. Currently available in the style file by uncommenting
\end{itemize}
\end{frame}
%#################################################




\end{document}

